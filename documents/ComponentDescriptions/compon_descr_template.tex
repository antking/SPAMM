\documentclass[12pt,letterpaper]{article}
%\usepackage{natbib}
%\usepackage{type1cm}
%\usepackage{epsf}
%\usepackage{graphicx}
%\usepackage{asp2006}
%stop double-spaced lists:
%\usepackage{mdwlist}
%\doublespace
\pagestyle{plain}

%If you look into the documentation, the attributes to \titlespacing are
%command, left margin, above-skip and below-skip respectively. The *
%notation replaces the formal notation using plus/minus and etc. If you
%set it to zero, headings will snug up to the paragraphs above and below
%them:
%\usepackage[compact]{titlesec}
%\titlespacing{\section}{0pt}{*2}{*1}
%\titlespacing{\subsection}{0pt}{*2}{*1}
%\titlespacing{\subsubsection}{0pt}{*1}{*1}

%%%%%%%%%%%%%%%%%%%%%%%%%%%%%%
	\oddsidemargin  0.0in
	\evensidemargin 0.0in
	\textwidth      6.5in
	\headheight     0.0in
	\topmargin      0.0in
	\textheight=9.0in
%%%%%%%%%%%%%%%%%%%%%%%%%%%%%%

\setlength{\parskip}{-1pt}
\setlength{\parsep}{0pt}
\setlength{\headsep}{0pt}
\setlength{\topskip}{0pt}
\setlength{\topmargin}{0pt}
\setlength{\topsep}{0pt}
\setlength{\partopsep}{0pt}

%\usepackage{fancyhdr}
%\pagestyle{fancy}
%\fancyhf{}
%\fancyfoot[C]{Project \thepage\ of 5}
%\renewcommand{\headrulewidth}{0pt}
%\renewcommand{\footrulewidth}{0pt}
%%\raggedbottom
%\raggedright
%\setlength{\tabcolsep}{0in}


\newcommand{\IUE}{{\it IUE}}
\newcommand{\HST}{{\it HST}}
\newcommand{\kms}{\ifmmode {\rm km\ s}^{-1} \else km s$^{-1}$\fi}
\newcommand{\Msun}{\ifmmode {\rm M}_{\odot} \else M$_{\odot}$\fi}
\newcommand{\Lsun}{\ifmmode {\rm L}_{\odot} \else L$_{\odot}$\fi}
\newcommand{\qo}{\ifmmode q_{\rm o} \else $q_{\rm o}$\fi}
\newcommand{\Ho}{\ifmmode H_{\rm o} \else $H_{\rm o}$\fi}
\newcommand{\ho}{\ifmmode h_{\rm o} \else $h_{\rm o}$\fi}
\newcommand{\ltsim}{\raisebox{-.5ex}{$\;\stackrel{<}{\sim}\;$}}
\newcommand{\gtsim}{\raisebox{-.5ex}{$\;\stackrel{>}{\sim}\;$}}
\newcommand{\vFWHM}{\ifmmode v_{\mbox{\tiny FWHM}} \else
                    $v_{\mbox{\tiny FWHM}}$\fi}
\newcommand{\CCF}{\ifmmode F_{\it CCF} \else $F_{\it CCF}$\fi}
\newcommand{\ACF}{\ifmmode F_{\it ACF} \else $F_{\it ACF}$\fi}
\newcommand{\Halpha}{\ifmmode {\rm H}\alpha \else H$\alpha$\fi}
\newcommand{\Hbeta}{\ifmmode {\rm H}\beta \else H$\beta$\fi}
\newcommand{\Hgamma}{\ifmmode {\rm H}\gamma \else H$\gamma$\fi}
\newcommand{\Hdelta}{\ifmmode {\rm H}\delta \else H$\delta$\fi}
\newcommand{\Lya}{\ifmmode {\rm Ly}\alpha \else Ly$\alpha$\fi}
\newcommand{\Lyb}{\ifmmode {\rm Ly}\beta \else Ly$\beta$\fi}
\newcommand{\HeI}{\ifmmode {\rm He}\,{\sc i}\,\lambda5876 \else 
	          He\,{\sc i}\,$\lambda5876$\fi}
\newcommand{\HeII}{\ifmmode {\rm He}\,{\sc ii}\,\lambda4686 \else 
	           He\,{\sc ii}\,$\lambda4686$\fi}
\newcommand{\hi}{H\,{\sc i}}
\newcommand{\hii}{H\,{\sc ii}}
\newcommand{\hei}{He\,{\sc i}}
\newcommand{\heii}{He\,{\sc ii}}
\newcommand{\fe}{Fe}
\newcommand{\feii}{Fe\,{\sc ii}}
\newcommand{\feiii}{Fe\,{\sc iii}}
\newcommand{\fevi}{Fe\,{\sc vi}}
\newcommand{\fevii}{Fe\,{\sc vii}}
\newcommand{\fex}{Fe\,{\sc x}}
\newcommand{\fexi}{Fe\,{\sc xi}}
\newcommand{\fexiv}{Fe\,{\sc xiv}}
\newcommand{\neiii}{Ne\,{\sc iii}}
\newcommand{\neiv}{Ne\,{\sc iv}}
\newcommand{\nev}{Ne\,{\sc v}}
\newcommand{\ci}{C\,{\sc i}}
\newcommand{\cii}{C\,{\sc ii}}
\newcommand{\ciii}{\ifmmode {\rm C}\,{\sc iii} \else C\,{\sc iii}\fi}
\newcommand{\civ}{C\,{\sc iv}}
\newcommand{\CIV}{\ifmmode {\rm C}\,{\sc iv}\,\lambda1549 \else 
	           C\,{\sc iv}\,$\lambda1549$\fi}
\newcommand{\Ni}{N\,{\sc i}}
\newcommand{\nii}{N\,{\sc ii}}
\newcommand{\niii}{N\,{\sc iii}}
\newcommand{\niv}{N\,{\sc iv}}
\newcommand{\nv}{N\,{\sc v}}
\newcommand{\oi}{O\,{\sc i}}
\newcommand{\oii}{O\,{\sc ii}}
\newcommand{\oiii}{O\,{\sc iii}}
\newcommand{\ob}{[O\,{\sc iii}]\,$\lambda \lambda 4959,5007$}
\newcommand{\oiv}{O\,{\sc iv}}
\newcommand{\ov}{O\,{\sc v}}
\newcommand{\ovi}{O\,{\sc vi}}
\newcommand{\mgi}{Mg\,{\sc i}}
\newcommand{\mgii}{Mg\,{\sc ii}}
\newcommand{\siiii}{Si\,{\sc iii}}
\newcommand{\Sizw}{Si\,{\sc ii}}
\newcommand{\siiv}{Si\,{\sc iv}}
\newcommand{\si}{S\,{\sc i}}
\newcommand{\sii}{S\,{\sc ii}}
\newcommand{\siii}{S\,{\sc iii}}
\newcommand{\caii}{Ca\,{\sc ii}}
\newcommand{\cav}{Ca\,{\sc v}}
\newcommand{\aliii}{Al\,{\sc iii}}
\newcommand{\sigbl}{$\sigma_{\rm blue}$}
\newcommand{\Flamunit}{erg s$^{-1}$\,cm$^{-2}$\,\AA$^{-1}$}
\newcommand{\lam}{$\lambda$}
\begin{document}

\begin{center}
{\large SPAMM -- Spectral Properties of AGN Modeled through MCMC }
\end{center}
\vspace{0.05in}

\section*{Spectral Components}
\subsection*{Nuclear Continuum}
\begin{equation}
F_{\lambda,{\rm PL}}=F_{\rm PL,0} \ (\frac{\lambda}{\lambda_0})^{\alpha} 
\end{equation}
where $F_{\rm PL,0}$ is the power-law normalization, $\alpha$ is the power-law slope and $\lambda_0$ is the median wavelength 
of the data wavelength range. 
\subsubsection*{Priors}

\subsection*{Balmer Continuum}
If we assume gas clouds with uniform temperature $\rm{T}_{\rm{e}}$, that are partially optically thick, for wavelengths bluer 
than the Balmer edge ($\lambda_{\rm{BE}}=3646$ \AA, rest frame), the Balmer spectrum can be parameterized as 
(Grandi et al., 1982): 
\begin{equation}
F_{\lambda,{\rm BC}}= F_{{\rm BE}} \ B_{\lambda}({\rm T}_{{\rm e}}) \ (1-{\rm e}^{-\tau_{{\rm BE}}(\frac{\lambda}{\lambda_{{\rm BE}}})^3}), \ \lambda<\lambda_{\rm{BE}}
\end{equation}
where $B_{\lambda}({\rm T}_{{\rm e}})$ is the Planck function at the electron temperature 
${\rm T}_{{\rm e}}$, $\tau_{{\rm BE}}$ is the optical 
depth at the Balmer edge, and $F_{{\rm BE}}$ is the normalized flux density at the Balmer edge.
\subsubsection*{Priors}

\subsection*{FeII \& FeIII}
Linear combination of N broadened and scaled iron templates:
\begin{equation}
F_{\lambda {\rm{,Fe}}} = \sum_{\rm i=1,..N} F_{\rm{Fe,0,i}} \  {\rm FeTempl}_{\rm{\lambda,i}}(\sigma_i) 
\end{equation}
where ${\rm FeTempl}_{\rm{\lambda,i}}$ is the iron template, $F_{\rm Fe,0,i}$ is the template normalization, and $\sigma_i$ is the 
width of the broadening kernel.
\subsubsection*{Priors}

\subsection*{Host Galaxy}
Linear combination of N galaxy templates:
\begin{equation}
F_{\lambda {\rm{,Host}}} = \sum_{\rm i=1,..N} F_{\rm{Host,0,i}} \  {\rm HostTempl}_{\rm{\lambda,i}} 
\end{equation}
where ${\rm HostTempl}_{\rm{\lambda,i}}$ is the host galaxy template, and $F_{\rm Host,0,i}$ is the template normalization.

\subsubsection*{Priors}

\subsection*{Host Galaxy Reddening}

\subsubsection*{Priors}

\subsection*{Nuclear Reddening}

\subsubsection*{Priors}

\subsection*{Emission lines}
\subsubsection*{Priors}


%\begin{center}
%{\Large This is my Project}
%\end{center}
%\vspace{0.1in}
%\begin{center}
%{\bf ABSTRACT}
%\end{center}




%\begin{figure}
%\plotone{figure.eps}
%\caption{}
%\end{figure}



\end{document}
