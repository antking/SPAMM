\documentclass[12pt,letterpaper]{article}
\usepackage{natbib}
\usepackage{layout}
\usepackage[british]{babel}
\usepackage[latin1]{inputenc}
\usepackage{amssymb}
\usepackage{textcomp}
%\usepackage{type1cm}
%\usepackage{epsf}
%\usepackage{graphicx}
%\usepackage{asp2006}
%stop double-spaced lists:
%\usepackage{mdwlist}
%\doublespace
\usepackage{hyperref}
\hypersetup{colorlinks=true}
\pagestyle{plain}

%If you look into the documentation, the attributes to \titlespacing are
%command, left margin, above-skip and below-skip respectively. The *
%notation replaces the formal notation using plus/minus and etc. If you
%set it to zero, headings will snug up to the paragraphs above and below
%them:
%\usepackage[compact]{titlesec}
%\titlespacing{\section}{0pt}{*2}{*1}
%\titlespacing{\subsection}{0pt}{*2}{*1}
%\titlespacing{\subsubsection}{0pt}{*1}{*1}

%%%%%%%%%%%%%%%%%%%%%%%%%%%%%%
	\oddsidemargin  0.0in
	\evensidemargin 0.0in
	\textwidth      6.5in
	\headheight     0.0in
	\topmargin      0.0in
	\textheight=9.0in
%%%%%%%%%%%%%%%%%%%%%%%%%%%%%%

\setlength{\parskip}{-1pt}
\setlength{\parsep}{0pt}
\setlength{\headsep}{0pt}
\setlength{\topskip}{0pt}
\setlength{\topmargin}{0pt}
\setlength{\topsep}{0pt}
\setlength{\partopsep}{0pt}

%\usepackage{fancyhdr}
%\pagestyle{fancy}
%\fancyhf{}
%\fancyfoot[C]{Project \thepage\ of 5}
%\renewcommand{\headrulewidth}{0pt}
%\renewcommand{\footrulewidth}{0pt}
%%\raggedbottom
%\raggedright
%\setlength{\tabcolsep}{0in}


\newcommand{\IUE}{{\it IUE}}
\newcommand{\HST}{{\it HST}}
\newcommand{\kms}{\ifmmode {\rm km\ s}^{-1} \else km s$^{-1}$\fi}
\newcommand{\Msun}{\ifmmode {\rm M}_{\odot} \else M$_{\odot}$\fi}
\newcommand{\Lsun}{\ifmmode {\rm L}_{\odot} \else L$_{\odot}$\fi}
\newcommand{\qo}{\ifmmode q_{\rm o} \else $q_{\rm o}$\fi}
\newcommand{\Ho}{\ifmmode H_{\rm o} \else $H_{\rm o}$\fi}
\newcommand{\ho}{\ifmmode h_{\rm o} \else $h_{\rm o}$\fi}
\newcommand{\ltsim}{\raisebox{-.5ex}{$\;\stackrel{<}{\sim}\;$}}
\newcommand{\gtsim}{\raisebox{-.5ex}{$\;\stackrel{>}{\sim}\;$}}
\newcommand{\vFWHM}{\ifmmode v_{\mbox{\tiny FWHM}} \else
                    $v_{\mbox{\tiny FWHM}}$\fi}
\newcommand{\CCF}{\ifmmode F_{\it CCF} \else $F_{\it CCF}$\fi}
\newcommand{\ACF}{\ifmmode F_{\it ACF} \else $F_{\it ACF}$\fi}
\newcommand{\Halpha}{\ifmmode {\rm H}\alpha \else H$\alpha$\fi}
\newcommand{\Hbeta}{\ifmmode {\rm H}\beta \else H$\beta$\fi}
\newcommand{\Hgamma}{\ifmmode {\rm H}\gamma \else H$\gamma$\fi}
\newcommand{\Hdelta}{\ifmmode {\rm H}\delta \else H$\delta$\fi}
\newcommand{\Lya}{\ifmmode {\rm Ly}\alpha \else Ly$\alpha$\fi}
\newcommand{\Lyb}{\ifmmode {\rm Ly}\beta \else Ly$\beta$\fi}
\newcommand{\HeI}{\ifmmode {\rm He}\,{\sc i}\,\lambda5876 \else 
	          He\,{\sc i}\,$\lambda5876$\fi}
\newcommand{\HeII}{\ifmmode {\rm He}\,{\sc ii}\,\lambda4686 \else 
	           He\,{\sc ii}\,$\lambda4686$\fi}
\newcommand{\hi}{H\,{\sc i}}
\newcommand{\hii}{H\,{\sc ii}}
\newcommand{\hei}{He\,{\sc i}}
\newcommand{\heii}{He\,{\sc ii}}
\newcommand{\fe}{Fe}
\newcommand{\feii}{Fe\,{\sc ii}}
\newcommand{\feiii}{Fe\,{\sc iii}}
\newcommand{\fevi}{Fe\,{\sc vi}}
\newcommand{\fevii}{Fe\,{\sc vii}}
\newcommand{\fex}{Fe\,{\sc x}}
\newcommand{\fexi}{Fe\,{\sc xi}}
\newcommand{\fexiv}{Fe\,{\sc xiv}}
\newcommand{\neiii}{Ne\,{\sc iii}}
\newcommand{\neiv}{Ne\,{\sc iv}}
\newcommand{\nev}{Ne\,{\sc v}}
\newcommand{\ci}{C\,{\sc i}}
\newcommand{\cii}{C\,{\sc ii}}
\newcommand{\ciii}{\ifmmode {\rm C}\,{\sc iii} \else C\,{\sc iii}\fi}
\newcommand{\civ}{C\,{\sc iv}}
\newcommand{\CIV}{\ifmmode {\rm C}\,{\sc iv}\,\lambda1549 \else 
	           C\,{\sc iv}\,$\lambda1549$\fi}
\newcommand{\alii}{Al\,{\sc ii}}
\newcommand{\Ni}{N\,{\sc i}}
\newcommand{\nii}{N\,{\sc ii}}
\newcommand{\niii}{N\,{\sc iii}}
%\newcommand{\niii]}{N\,{\sc iii}]\fi}
\newcommand{\niv}{N\,{\sc iv}}
\newcommand{\nv}{N\,{\sc v}}
\newcommand{\oi}{O\,{\sc i}}
\newcommand{\oii}{O\,{\sc ii}}
\newcommand{\oiii}{O\,{\sc iii}}
\newcommand{\ob}{[O\,{\sc iii}]\,$\lambda \lambda 4959,5007$}
\newcommand{\oiv}{O\,{\sc iv}}
\newcommand{\ov}{O\,{\sc v}}
\newcommand{\ovi}{O\,{\sc vi}}
\newcommand{\mgi}{Mg\,{\sc i}}
\newcommand{\mgii}{Mg\,{\sc ii}}
\newcommand{\siiii}{Si\,{\sc iii}}
\newcommand{\siiiifb}{[Si\,{\sc iii}]}
\newcommand{\Sizw}{Si\,{\sc ii}}
\newcommand{\siiv}{Si\,{\sc iv}}
\newcommand{\si}{S\,{\sc i}}
\newcommand{\sii}{S\,{\sc ii}}
\newcommand{\siii}{S\,{\sc iii}}
\newcommand{\caii}{Ca\,{\sc ii}}
\newcommand{\cav}{Ca\,{\sc v}}
\newcommand{\aliii}{Al\,{\sc iii}}
\newcommand{\sigbl}{$\sigma_{\rm blue}$}
\newcommand{\Flamunit}{erg s$^{-1}$\,cm$^{-2}$\,\AA$^{-1}$}
\newcommand{\lam}{$\lambda$}
\begin{document}

\begin{center}
{\large SPAMM -- Spectral Properties of AGN Modeled through MCMC }
\end{center}
\vspace{0.05in}

\section*{Spectral Components}

\subsection*{Nuclear Continuum}
\begin{equation}
F_{\lambda,{\rm PL}}=F_{\rm PL,0} \ \left(\frac{\lambda}{\lambda_0}\right)^{\alpha} 
\end{equation}
where $F_{\rm PL,0}$ is the power-law normalization, $\alpha$ is the power-law slope and $\lambda_0$ is the median wavelength 
of the observed wavelength range. 

\subsubsection*{Code Parameters}
\begin{itemize}
    \item {\tt param1}: power-law slope ($\alpha_{\lambda}$)
    \item {\tt param2}: power-law normalization ($F_{\rm PL,0}$)
\end{itemize}

\subsubsection*{Priors}
\begin{itemize}
    \item {\tt $\alpha_{\lambda}$}: flat prior in range [-3,3]
    \item {\tt $F_{\rm PL,0}$}: flat prior between 0 and the maximum of the spectral flux after computing running median \textit{(to be discussed)} 
\end{itemize}


\subsection*{Balmer Continuum}
If we assume gas clouds with uniform temperature $\rm{T}_{\rm{e}}$, that are partially optically thick, for wavelengths bluer 
than the Balmer edge ($\lambda_{\rm{BE}}=3646$ \AA, rest frame), the Balmer spectrum can be parametrized as 
(Grandi et al., 1982): 
\begin{equation}
F_{\lambda,{\rm BC}}= F_{{\rm BE}} \ B_{\lambda}\left({\rm T}_{{\rm e}}\right) \ \left(1-{\rm e}^{-\tau_{{\rm BE}}\left(\frac{\lambda}{\lambda_{{\rm BE}}}\right)^3}\right), \ \lambda<\lambda_{\rm{BE}}
\end{equation}
where $\rm{B}_{\lambda}(\rm{T}_{\rm{e}})$ is the Planck function at the electron temperature $\rm{T}_{\rm{e}}$, $\tau_{\rm{BE}}$ is the optical 
depth at the Balmer edge, and $\rm{F}_{\rm{BE}}$ is the normalized flux density at the Balmer edge. 
At wavelengths $\lambda>$3646 \AA \ high order Balmer lines are merging into a pseudo continuum, yielding a smooth rise to the Balmer edge. 

\textit{Discussion: there is the possibility of supplementing the functional form for $\lambda>$3646 \AA \ with high-order Balmer emission lines (up to n=50), 
that might be computed using the relative line strengths as given by Storey 1995, case B. Possibly not needed if we are modeling continuum and emission lines 
(including high-order Balmer lines) at the same time.}

\subsubsection*{Code Parameters}
\begin{itemize}
    \item {\tt param1}: electron temperature (${\rm T}_{{\rm e}}$)
    \item {\tt param2}: optical depth at the Balmer edge ($\tau_{\rm{BE}}$)
    \item {\tt param3}: normalized flux density at the Balmer edge ($\rm{F}_{\rm{BE}}$)
\end{itemize}

\subsubsection*{Priors}
\begin{itemize}
    \item {\tt ${\rm T}_{{\rm e}}$}: flat prior in the [5,000-20,000] Kelvin range.
    \item {\tt $\tau_{\rm{BE}}$}: flat prior in the [0.1-2.0] range.    
    \item {\tt $\rm{F}_{\rm{BE}}$}:
      \begin{itemize} 
	\item {Option 1}: if the Balmer edge is included in the spectral range, flat prior between 0 and the observed spectral flux at the Balmer edge after computing 
	  running median \textit{(to be discussed)}
	\item {Option 2}: if the Balmer edge is not included in the spectral range, flat prior between 0 and the maximum flux after computing running median 
	  \textit{(super conservative, to be discussed)}
      \end{itemize}
\end{itemize}

\subsection*{FeII \& FeIII}
Linear combination of N broadened and scaled iron templates:
\begin{equation}
F_{\lambda {\rm{,Fe}}} = \sum_{\rm i=1,..N} F_{\rm{Fe,0,i}} \  {\rm FeTempl}_{\rm{\lambda,i}}(\sigma_i) 
\end{equation}		
where ${\rm FeTempl}_{\rm{\lambda,i}}$ is the iron template, $F_{\rm Fe,0,i}$ is the template normalization, 
and $\sigma_i$ is the width of the broadening kernel.

\subsubsection*{Code Parameters}
\begin{itemize}
    \item {\tt param1}: iron templates (${\rm FeTempl}_{\rm{\lambda,i}}$)
    \begin{itemize}
       \item {\tt UV template}: Vestergaard \& Wilkes (2001), (1250-3090) \AA
       \item {\tt Optical template}: V\'{e}ron-Cetty et al. (2004), (3535-7530) \AA
       \item {\tt Gap between UV and Optical template}: Beverly Wills, (3090-3534.4) \AA \ \textit{(In Marianne's hands)}
       \item {\tt} \textit{Discussion: more options?} 
    \end{itemize}
    \item {\tt param2}: width of the broadening Lorentzian kernel ($\sigma_i$)
    \item {\tt param3}: template normalization ($F_{\rm Fe,0,i}$)  
\end{itemize}

\subsubsection*{Priors}
\begin{itemize}
    \item {\tt $\sigma_i$}: flat prior in the [500-20,000] km/s range. The other possibility is to have a Gaussian prior centered on the line width of \Hbeta.
    \item {\tt $F_{\rm Fe,0,i}$}: flat prior between 0 and the maximum flux after computing running median \textit{(super conservative, to be discussed)}
\end{itemize}

\subsection*{Host Galaxy}
Linear combination of N galaxy templates:
\begin{equation}
F_{\lambda {\rm{,Host}}} = \sum_{\rm i=1,..N} F_{\rm{Host,0,i}} \  {\rm HostTempl}_{\rm{\lambda,i}} 
\end{equation}
where ${\rm HostTempl}_{\rm{\lambda,i}}$ is the host galaxy template, and $F_{\rm Host,0,i}$ is the template normalization.\\

\noindent Possible useful codes for inspection: \\
\begin{tabular}{ l | c  || l }
Code & Paper & Comments\\ \hline
\htmladdnormallink{STARLIGHT}{http://www.starlight.ufsc.br} & Cid Fernandes, R. et al. 2005, MNRAS, 358, 363 & One spectrum at a time\\
\htmladdnormallink{GANDALF}{http://star-www.herts.ac.uk/~sarzi/PaperV_nutshell/PaperV_nutshell.html} & Sarzi et al. 2006, MNRAS, 366, 1151 & Deals with 2D data\\ \hline
\end{tabular}

\subsubsection*{Code Parameters}
\begin{itemize}
    \item {\tt param1}: Synthesis model: Choice of templates to combine
    \begin{itemize}
       \item {\tt Option 1}: Observed Template Galaxies (e.g., Kinney et al. 1996)
       \item {\tt Option 2}: Observed Template Stars
       \item {\tt Option 3}: \href{http://adsabs.harvard.edu/abs/1997A&A...326..950F} {P\'{E}GASE}
       \item {\tt Option 4}: \href{http://www.stsci.edu/science/starburst99/docs/default.htm}{Starburst99}
       \item {\tt Option 5}: \href{http://www2.iap.fr/users/charlot/bc2003/}{Bruzal \& Charlot 2003}
    \end{itemize}
    \item {\tt param2}: How many models/templates to include
    \item {\tt param3}: Kinematics -- fit or fix?
    \item {\tt param4}: Wavelength Range
    \item {\tt param5}: Mask -- block certain wavelength ranges e.g., H$\alpha$
\end{itemize}

\subsubsection*{Priors}
  \begin{enumerate}
  	\item Age range of interest?
	\item Metallicity range?
	\item redshift
  \end{enumerate}

\subsection*{Host Galaxy Reddening}

\subsubsection*{Code Parameters}
\begin{itemize}
    \item {\tt param1}: Possible reddening laws:
    \begin{itemize}
      \item {\tt Option 1}: Milky Way
      \item {\tt Option 2}: Large Magellanic Cloud, LMC
      \item {\tt Option 3}: Small Magellanic Cloud, SMC
      \item {\tt Option 4}: Fit for $R_v$?
    \end{itemize}
    \item {\tt param2}: Dust\_geometry: foreground screen or mixed media.
\end{itemize}

\subsubsection*{Priors}
  \begin{itemize}
   \item {\tt $\tau_{\nu}$}: flat between zero and 1.0 \textit{Discussion: do we need higher $\tau$ values?}
  \end{itemize}

\subsection*{Nuclear Reddening}

\subsubsection*{Code Parameters}
\begin{itemize}
    \item {\tt param1}: Possible reddening laws:
    \begin{itemize}
      \item {\tt Option 1}: Small Magellanic Cloud, SMC
      \item {\tt Option 2}: Fit for $R_v$?
    \end{itemize}
    \item {\tt param2}: Dust\_geometry: foreground screen or mixed media.
\end{itemize}

\subsubsection*{Priors}
  \begin{itemize}
   \item {\tt $\tau_{\nu}$}: flat between zero and 1.0 \textit{Discussion: do we need higher $\tau$ values?}
  \end{itemize}


\subsection*{Emission lines}
Functional fitting to broad and narrow emission-line components.

\begin{itemize}
  \item {\bf Broad Emission Line List, \boldmath{$\lambda_{0,b}$}}
    \begin{itemize}
      \itemsep-0.1cm
      \item \Lya\, $\lambda$1215 (actual $\lambda=1215.670$\AA)
      \item \nv\, $\lambda$1240 (doublet at $\lambda \lambda$1238.808, 1242.796\AA)
      \item ``1400 Feature'': \siiv\ (doublet at $\lambda \lambda$1393.755, 1402.770\AA) plus \oiv] blend ($\lambda \lambda$1397.210, 1399.780, 1404.790, 1407.390\AA)
      \item \niv]\, $\lambda$1486 (actual $\lambda=1486.500$\AA)
      \item \CIV\ (unresolved doublet at $\lambda \lambda$1548.188, 1550.762\AA) 
      \item \heii\, $\lambda$1640 (actual $\lambda=1640.720$\AA)
      \item \oiii] $\lambda$1663 (doublet at $\lambda \lambda$1660.800, 1666.140\AA)
      \item \ciii]\, $\lambda$1909: actually a blend of \aliii\, $\lambda \lambda$1854.720, 1862.780\AA, \siiii]\, $\lambda=1892.030$\AA, and \ciii]\, $\lambda=1908.734$\AA.
      \item \mgii\, $\lambda$2798 (doublet at $\lambda \lambda$2796.350, 2803.530\AA)
      \item \Hdelta\ $\lambda=4101.735$\AA
      \item \Hgamma\ $\lambda=4340.450$\AA
      \item \HeII\ (actual $\lambda=4685.650$\AA) 
      \item \Hbeta\ $\lambda=4861.320$\AA
      \item \hei\, $\lambda$4922 (actual $\lambda=4921.9$\AA)
      \item \hei\ $\lambda=5016$\AA 
      \item \hei\, $\lambda$5876 (actual $\lambda=5875.680$\AA) 
      \item \hei\, $\lambda$6678 (actual $\lambda=6678.000$\AA) 
      \item \hei\, $\lambda$7065 (actual $\lambda=7065.300$\AA) 
      \item \Halpha\ $\lambda=6562.780$\AA
    \end{itemize}

\textit{Comment: MV has a long list of Helium and Balmer lines in the optical and UV from H$\alpha$ to the Balmer Jump 
that goes to high order (\#50 in Balmer line series). The list also has relative line ratios and relative amplitude of 
BaC jump.}

\item{\bf Narrow Emission Line List, \boldmath{$\lambda_{0,n}$}}
  \begin{itemize}
    \itemsep-0.1cm
    \item {[\nev]\, $\lambda$3425.900\AA}
    \item {[\oii]\, $\lambda \lambda$3726.000, 3728.800\AA}
    \item {[\neiii]\, $\lambda$3868.800\AA}
    \item \heii\ (actual $\lambda=4685.650$\AA) 
    \item \Hbeta\ $\lambda=4861.320$\AA
    \item {[\oiii]\, $\lambda \lambda$4958.920, 5006.850\AA}
    \item {[\nii]\, $\lambda \lambda$6548.060, 6583.39\AA}
    \item \Halpha\ $\lambda=6562.780$\AA
    \item {[\Sizw]\, $\lambda \lambda$6716.420, 6730.780\AA}
  \end{itemize}

\end{itemize}

\subsubsection*{Fitting Function Possibilities}

  \begin{itemize}
    \item Narrow Lines
      \begin{itemize}
        \item Single Gaussian with Prior (1) 
	\item Double Gaussian with Prior (1)  
        \item Option (automatically test) for additional (broader) Gaussian to \ob\ base. 
      \end{itemize}
    \item Broad Lines
      \begin{itemize}
        \item Multiple Gaussians
        \item Multiple Gauss-Hermite polynomials
        \item Gaussian (very broad) plus Gauss-Hermite (broad)
        \item Multiple Lorentzians
        \item Mix of Gaussian and Lorentzian(s) (i.e., Voigt profile)
	\item Powerlaw profiles + 1-2 Gaussians 
      \end{itemize}
   \end{itemize}

\subsubsection*{Functional Forms}

  \begin{itemize}
    \item Gaussian:
      \begin{equation} F_{\lambda} = \frac{f_{\rm peak}}{\sigma \sqrt{2\pi}}e^{-\frac{1}{2}\left(\frac{\lambda - \mu}{\sigma}\right)^2},
        \end{equation}
        where the Gaussian FWHM$=2\sqrt{2\ln 2}\sigma$ and $\mu=\lambda_0$(broad, narrow).  \\ Free parameters: $f_{\rm peak}$ (peak flux), $\mu$, $\sigma$.
	For multiple Gaussian components, the amplitudes can be tied relative to one another (i.e. to the amplitude of the first component).

	\textit{Discussion: MV suggests to use relative velocity shift rather than wavelengths shift, since it is more physical}

     \item $6{\rm th}$ Order Gauss-Hermite Polynomial:
       \begin{equation} F_{\lambda} = [f_{\rm peak} \alpha(w)/\sigma]\left(1 + \sum_{j=3}^{6}h_jH_j(w) \right), 
       \end{equation} 
       \begin{equation} w\equiv (\lambda - \mu)/\sigma,
       \end{equation}
       \begin{equation} \alpha(w) = \frac{1}{2\sqrt{\pi}} e^{-\frac{1}{2}w^2}. \end{equation}
       where this follows the normalization of van der Marel \& Franx (1993, ApJ, 407, 525; first equation).  The $H_j$ coefficients can be found in Cappellari et al.\ (2002, ApJ, 578, 787):
       \begin{equation} H_3(w) = \frac{w(2w^2-3)}{\sqrt{3}}, \end{equation}
       \begin{equation} H_4(w) = \frac{w^2(4w^2-12)+3}{2\sqrt{6}}, \end{equation}
       \begin{equation} H_5(w) = \frac{w[w^2(4w^2-20)+15]}{2\sqrt{15}}, \end{equation}
       \begin{equation} H_6(w) = \frac{w^2[w^2(8w^2-60)+90]-15}{12\sqrt{5}}. \end{equation}
       Free parameters: $f_{\rm peak}$, $\mu$, $\sigma$, $h_3$, $h_4$, $h_5$, $h_6$.
     \item Lorentzian
       \begin{equation} F_{\lambda} = \frac{f_{\rm peak}}{\pi} \frac{\frac{1}{2}\sigma}{(\lambda - \mu)^2+(\frac{1}{2}\sigma)^2},
       \end{equation}
       where $\mu=\lambda_0$(b,n) and the Lorentzian FWHM = $\sigma = 2f_{\rm peak}/(\pi F(\mu))$.
     \item Powerlaw profile:
  \end{itemize}

\subsubsection*{Priors}
  \begin{enumerate}
    \item Limit all component positions (i.e., velocity offset from laboratory wavelengths) to within a given wavelength (or velocity) range to prevent the components 
	  to wander.
    \item Width and velocity shifts of each of the Gaussian components of narrow forbidden lines tied together and FWHM $<$1200 km s$^{-1}$
    \item Ranges of widths and velocity shifts to be included. I.e., profile limits to 
          be specified - either one for each emission line, or for each type of line 
          (broad, narrow, weak, strong, etc.) \textit{Comment: MV has a separate long list - don't want to list here in case it needs to 
           be coded differently.}
    \item Narrow emission line redshift solution, i.e., $\mu = \lambda_{0,n}(1+z)\pm \Delta \mu$ is constant.
    \item Narrow line doublet ratios fixed:
      \begin{itemize}
        \item {[\oii]\, $\lambda \lambda$3726.000, 3728.800\AA; ??:?? (This is density dependent)}
        \item {[\oiii]\, $\lambda \lambda$4958.920, 5006.850\AA;  1:3}
        \item {[\Sizw]\, $\lambda \lambda$6716.420, 6730.780\AA; ??:??}
        \item {[\nii]\, $\lambda \lambda$6548.060, 6583.39\AA; ??:??}
      \end{itemize}
    \item Fix relative line ratios of all Balmer lines 
    \item Fluxes must be non-negative (BLR and NLR emission)
    \item Tie together the widths and velocity shifts of broad line components of identical species, e.g., \heii\, $\lambda$1640 and \HeII?  \textit{(To be discussed)}.
    \item assumptions about CIV redshelf?? Additional HeII component? He\,II, \feii, Al\,III, O\,II].
    \item Suggested Parameter Space to search \textit{(to be discussed)}
      \begin{itemize}
        \item $f_{\rm peak}/f_{\rm cont}$ = [0, 1.d4, 1.d-3]
        \item $\mu = \lambda_{0,n}(1+z)\pm 1000\, {\rm km}\, {\rm s}^{-1}; \Delta \mu \sim f$(pixscale)
        \item $\sigma$ = [100, 3.d4]; $\Delta \sigma \sim f$(pixscale)
        \item $h_j$ = [-0.3, 0.3, 1.d-3]
      \end{itemize}
  \end{enumerate}

\subsubsection*{Code Parameters}
\begin{itemize}
    \item {\tt param1}: description of parameter 1 here
\end{itemize}

\textit{Discussion: Do we want to measure line dispersion on functional fit to the data or to the residual data (after eliminating the modeled blending line emission from 
other emission contributions)?}



%\begin{center}
%{\Large This is my Project}
%\end{center}
%\vspace{0.1in}
%\begin{center}
%{\bf ABSTRACT}
%\end{center}




%\begin{figure}
%\plotone{figure.eps}
%\caption{}
%\end{figure}



\end{document}
